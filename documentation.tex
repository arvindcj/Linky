% Created 2017-05-27 Sat 21:50
\documentclass[11pt]{article}
\usepackage[utf8]{inputenc}
\usepackage[T1]{fontenc}
\usepackage{fixltx2e}
\usepackage{graphicx}
\usepackage{longtable}
\usepackage{float}
\usepackage{wrapfig}
\usepackage{rotating}
\usepackage[normalem]{ulem}
\usepackage{amsmath}
\usepackage{textcomp}
\usepackage{marvosym}
\usepackage{wasysym}
\usepackage{amssymb}
\usepackage{hyperref}
\tolerance=1000
\author{Arvind Janakiram Cuthambakam - 217007335 \\ Author 1 \\ etc}
\date{\today}
\title{SIT782 Assignemnt 3 (System Implementation)}
\hypersetup{
  pdfkeywords={},
  pdfsubject={},
  pdfcreator={Emacs 25.1.1 (Org mode 8.2.10)}}
\begin{document}

\maketitle
\tableofcontents


\section{Description of System Developed}
\label{sec-1}

\subsection{What is the system Developed?}
\label{sec-1-1}
The system being developed is a Text-based web browser.

\subsection{Functionalities and the features of your system?}
\label{sec-1-2}
\begin{itemize}
\item Text-based web browser.
\item HTTP Client.
\item HTML Parser.
\item HTML Renderer.
\item Mouse Event Listener.
\item Interactive Graphical User-Interface.
\end{itemize}

\subsection{Comparison to your proposal. State what was completed and what has not and why?}
\label{sec-1-3}
\subsubsection{Completed Features}
\label{sec-1-3-1}
\begin{itemize}
\item HTTP Client: it is the module that is responsible for fetching the HTML document form the remote web-server. The browser works on HTTP GET Requests.
\item Mouse Event Listener: the module listens was built to intercept users-inputs and trigger corresponding events.
\item Interactive Graphical User-Interface: the user interface is very simple, with an input for web URl. Once when the user enters the URL and hits enter the HTML is rendered to the screen.
\item HTML Parser: the parser parses the HTML from the HTTP response into a parse tree.
\item HTML Renderer: the renderer parses the HTML into the final display. Although, only a small but the most popular HTML Tags are handled. Tags handled COMMENTS, HTML, HEAD, BODY, DIV, TITLE, P and H1 - H6.
\end{itemize}

Although, all the features mentioned above were implemented, certain features were developed to a point to demoninstrate its functionality. Mordern browsers have been around from 1990 and have built up very sophisticated features. The aim for the current project was to demoninstrate a skeleton implementation within a semester. 

\subsubsection{In-Complete Features}
\label{sec-1-3-2}
\begin{itemize}
\item There are 120 HTML tags defined by the W3c.
\item Out of the above 10 Tags were depreciated.
\item There are about 25 HTML Tags most commonly used.
\item Out of 25 of the most popular TAGS used 10 were supported.
\item The 10 Tags supported demoninstrated the key principles used in constructing a web browser.
\end{itemize}

\subsection{Brief Summary on Achievement overall and each individual}
\label{sec-1-4}
\subsubsection{Building a web browser from scratch requires large technical challenges. Some of the channenges were:}
\label{sec-1-4-1}
\begin{itemize}
\item Limited to 10 development hours per week for the entire project.
\item Lack of design documentation for existing projects and literature in the subject.
\item Most of the programming involved was experimental.
\item Compilers were studied to understand parsing and rendering.
\end{itemize}
\subsubsection{Individual Contributions}
\label{sec-1-4-2}
\begin{enumerate}
\item Arvind
\label{sec-1-4-2-1}
\begin{itemize}
\item Design and Analysis of the project.
\item Build HTML Parser and Renderer.
\end{itemize}
\end{enumerate}

\section{System Implementation Details}
\label{sec-2}
\subsection{What has been Implemented?}
\label{sec-2-1}
\begin{itemize}
\item HTTP Client.
\item HTML Parser.
\item HTML Renderer.
\item Mouse Event Listener.
\item Interactive Graphical User-Interface.
\end{itemize}

\subsection{How was it Implemented?}
\label{sec-2-2}
\subsubsection{Techniques, languages, tools?}
\label{sec-2-2-1}
\begin{itemize}
\item Java
\item Gradle
\item JSoup Library
\end{itemize}

\subsubsection{Algorithms}
\label{sec-2-2-2}
\begin{itemize}
\item Custom Model-View-Controller design pattern.
\item Singleton Degign Pattern for the renderer.
\item Factory Design Pattern for the RendererObject.
\item Tree based algorithms for the Parser and Renderer.
\end{itemize}

\subsubsection{Tools}
\label{sec-2-2-3}
\begin{itemize}
\item Gradle was used for build management.
\item Trello was used for project management.
\item Git was used to version code.
\end{itemize}


\section{System Testing and Quality Assurance}
\label{sec-3}
\subsection{Test Plan}
\label{sec-3-1}
\subsubsection{Test Objective/ Requirement}
\label{sec-3-1-1}
\begin{itemize}
\item Iterative and experimental development required that tests be performed directly on the implementation.
\item Testing involved White, Black and Grey box testing whith minimal or no Unit tests.
\end{itemize}
\subsubsection{Assumptions}
\label{sec-3-1-2}
\begin{itemize}
\item Performance and Security Testing Out-of-scope.
\item No automated tests were considered as the project was experimental.
\end{itemize}


\section{System Documentation}
\label{sec-4}
\subsection{User Manual}
\label{sec-4-1}
% Emacs 25.1.1 (Org mode 8.2.10)
\end{document}